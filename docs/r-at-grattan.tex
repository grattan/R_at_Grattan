\documentclass[]{book}
\usepackage{lmodern}
\usepackage{amssymb,amsmath}
\usepackage{ifxetex,ifluatex}
\usepackage{fixltx2e} % provides \textsubscript
\ifnum 0\ifxetex 1\fi\ifluatex 1\fi=0 % if pdftex
  \usepackage[T1]{fontenc}
  \usepackage[utf8]{inputenc}
\else % if luatex or xelatex
  \ifxetex
    \usepackage{mathspec}
  \else
    \usepackage{fontspec}
  \fi
  \defaultfontfeatures{Ligatures=TeX,Scale=MatchLowercase}
\fi
% use upquote if available, for straight quotes in verbatim environments
\IfFileExists{upquote.sty}{\usepackage{upquote}}{}
% use microtype if available
\IfFileExists{microtype.sty}{%
\usepackage{microtype}
\UseMicrotypeSet[protrusion]{basicmath} % disable protrusion for tt fonts
}{}
\usepackage{hyperref}
\hypersetup{unicode=true,
            pdftitle={Using R at Grattan Institute},
            pdfauthor={Will Mackey and Matt Cowgill},
            pdfborder={0 0 0},
            breaklinks=true}
\urlstyle{same}  % don't use monospace font for urls
\usepackage{natbib}
\bibliographystyle{apalike}
\usepackage{color}
\usepackage{fancyvrb}
\newcommand{\VerbBar}{|}
\newcommand{\VERB}{\Verb[commandchars=\\\{\}]}
\DefineVerbatimEnvironment{Highlighting}{Verbatim}{commandchars=\\\{\}}
% Add ',fontsize=\small' for more characters per line
\usepackage{framed}
\definecolor{shadecolor}{RGB}{248,248,248}
\newenvironment{Shaded}{\begin{snugshade}}{\end{snugshade}}
\newcommand{\AlertTok}[1]{\textcolor[rgb]{0.94,0.16,0.16}{#1}}
\newcommand{\AnnotationTok}[1]{\textcolor[rgb]{0.56,0.35,0.01}{\textbf{\textit{#1}}}}
\newcommand{\AttributeTok}[1]{\textcolor[rgb]{0.77,0.63,0.00}{#1}}
\newcommand{\BaseNTok}[1]{\textcolor[rgb]{0.00,0.00,0.81}{#1}}
\newcommand{\BuiltInTok}[1]{#1}
\newcommand{\CharTok}[1]{\textcolor[rgb]{0.31,0.60,0.02}{#1}}
\newcommand{\CommentTok}[1]{\textcolor[rgb]{0.56,0.35,0.01}{\textit{#1}}}
\newcommand{\CommentVarTok}[1]{\textcolor[rgb]{0.56,0.35,0.01}{\textbf{\textit{#1}}}}
\newcommand{\ConstantTok}[1]{\textcolor[rgb]{0.00,0.00,0.00}{#1}}
\newcommand{\ControlFlowTok}[1]{\textcolor[rgb]{0.13,0.29,0.53}{\textbf{#1}}}
\newcommand{\DataTypeTok}[1]{\textcolor[rgb]{0.13,0.29,0.53}{#1}}
\newcommand{\DecValTok}[1]{\textcolor[rgb]{0.00,0.00,0.81}{#1}}
\newcommand{\DocumentationTok}[1]{\textcolor[rgb]{0.56,0.35,0.01}{\textbf{\textit{#1}}}}
\newcommand{\ErrorTok}[1]{\textcolor[rgb]{0.64,0.00,0.00}{\textbf{#1}}}
\newcommand{\ExtensionTok}[1]{#1}
\newcommand{\FloatTok}[1]{\textcolor[rgb]{0.00,0.00,0.81}{#1}}
\newcommand{\FunctionTok}[1]{\textcolor[rgb]{0.00,0.00,0.00}{#1}}
\newcommand{\ImportTok}[1]{#1}
\newcommand{\InformationTok}[1]{\textcolor[rgb]{0.56,0.35,0.01}{\textbf{\textit{#1}}}}
\newcommand{\KeywordTok}[1]{\textcolor[rgb]{0.13,0.29,0.53}{\textbf{#1}}}
\newcommand{\NormalTok}[1]{#1}
\newcommand{\OperatorTok}[1]{\textcolor[rgb]{0.81,0.36,0.00}{\textbf{#1}}}
\newcommand{\OtherTok}[1]{\textcolor[rgb]{0.56,0.35,0.01}{#1}}
\newcommand{\PreprocessorTok}[1]{\textcolor[rgb]{0.56,0.35,0.01}{\textit{#1}}}
\newcommand{\RegionMarkerTok}[1]{#1}
\newcommand{\SpecialCharTok}[1]{\textcolor[rgb]{0.00,0.00,0.00}{#1}}
\newcommand{\SpecialStringTok}[1]{\textcolor[rgb]{0.31,0.60,0.02}{#1}}
\newcommand{\StringTok}[1]{\textcolor[rgb]{0.31,0.60,0.02}{#1}}
\newcommand{\VariableTok}[1]{\textcolor[rgb]{0.00,0.00,0.00}{#1}}
\newcommand{\VerbatimStringTok}[1]{\textcolor[rgb]{0.31,0.60,0.02}{#1}}
\newcommand{\WarningTok}[1]{\textcolor[rgb]{0.56,0.35,0.01}{\textbf{\textit{#1}}}}
\usepackage{longtable,booktabs}
\usepackage{graphicx,grffile}
\makeatletter
\def\maxwidth{\ifdim\Gin@nat@width>\linewidth\linewidth\else\Gin@nat@width\fi}
\def\maxheight{\ifdim\Gin@nat@height>\textheight\textheight\else\Gin@nat@height\fi}
\makeatother
% Scale images if necessary, so that they will not overflow the page
% margins by default, and it is still possible to overwrite the defaults
% using explicit options in \includegraphics[width, height, ...]{}
\setkeys{Gin}{width=\maxwidth,height=\maxheight,keepaspectratio}
\IfFileExists{parskip.sty}{%
\usepackage{parskip}
}{% else
\setlength{\parindent}{0pt}
\setlength{\parskip}{6pt plus 2pt minus 1pt}
}
\setlength{\emergencystretch}{3em}  % prevent overfull lines
\providecommand{\tightlist}{%
  \setlength{\itemsep}{0pt}\setlength{\parskip}{0pt}}
\setcounter{secnumdepth}{5}
% Redefines (sub)paragraphs to behave more like sections
\ifx\paragraph\undefined\else
\let\oldparagraph\paragraph
\renewcommand{\paragraph}[1]{\oldparagraph{#1}\mbox{}}
\fi
\ifx\subparagraph\undefined\else
\let\oldsubparagraph\subparagraph
\renewcommand{\subparagraph}[1]{\oldsubparagraph{#1}\mbox{}}
\fi

%%% Use protect on footnotes to avoid problems with footnotes in titles
\let\rmarkdownfootnote\footnote%
\def\footnote{\protect\rmarkdownfootnote}

%%% Change title format to be more compact
\usepackage{titling}

% Create subtitle command for use in maketitle
\providecommand{\subtitle}[1]{
  \posttitle{
    \begin{center}\large#1\end{center}
    }
}

\setlength{\droptitle}{-2em}

  \title{Using R at Grattan Institute}
    \pretitle{\vspace{\droptitle}\centering\huge}
  \posttitle{\par}
    \author{Will Mackey and Matt Cowgill}
    \preauthor{\centering\large\emph}
  \postauthor{\par}
      \predate{\centering\large\emph}
  \postdate{\par}
    \date{2019-09-07}

\usepackage{booktabs}
\usepackage{amsthm}
\makeatletter
\def\thm@space@setup{%
  \thm@preskip=8pt plus 2pt minus 4pt
  \thm@postskip=\thm@preskip
}
\makeatother

\begin{document}
\maketitle

{
\setcounter{tocdepth}{1}
\tableofcontents
}
\hypertarget{welcome}{%
\chapter*{Welcome}\label{welcome}}
\addcontentsline{toc}{chapter}{Welcome}

This guide is designed for everyone who uses -- or would like to use -- R at Grattan Institute.

It does two main things:

\begin{enumerate}
\def\labelenumi{\arabic{enumi}.}
\tightlist
\item
  Shows you how to use R to complete common analytical tasks you'll face at Grattan.
\item
  Sets out some guidelines and good practices when using R at Grattan.
\end{enumerate}

As a guide to using R, this website is helpful but incomplete. We can't possibly cover - or anticipate - all the skills you might need to know. If you make it to the end of this guide and want to learn more, start by reading \href{https://r4ds.had.co.nz}{R for Data Science} by Hadley Wickham and Garrett Grolemund. It's free.

Any complaints or comments about this guide can be sent to Matt or Will, respectively.

\hypertarget{introduction-to-r}{%
\chapter{Introduction to R}\label{introduction-to-r}}

Most people reading this guide will know what R is. But if you don't - that's OK!

If you have used R before and are comfortable enough with it, you might want to skip to the next page. This page is intended for people who are unfamiliar with R.

\hypertarget{what-is-r}{%
\section{What is R?}\label{what-is-r}}

R is a programming language that is designed by and for statisticians, data scientists, and other people who work with data. It's free - you can download R at no charge. It's also open source - you can view and (if you're game) modify the code that underlies the R language. R is available for all major computing platforms including Windows, macOS, and Linux.

R has a lot in common with other statistical software like SAS, Stata, SPSS or Eviews. You can use those software packages to read data, manipulate it, generate summary statistics, estimate models, and so on. You can use R for all those things and more. You interact with R by writing code. This is a little different to Stata or SPSS, which allow you to do at least part of your analyses by clicking on menus and buttons. This means the initial learning curve for R can be a little steeper than for something like SPSS, but there are great benefits to a code-based approach to data analysis (\protect\hyperlink{why-script}{see the next page for more on this}).

R also has some overlap with general purpose programming languages like Python. But R is more focused on the sort of tasks that statisticians, data scientists, and academic researchers do.

R is quite old, having been first released publicly in 1995, but it's also growing and changing rapidly. A lot of developments in R come in the form of new add-on pieces of software - known as `packages' - that extend R's functionality in some way. We cover packages more \protect\hyperlink{packages}{later in this page}.

When you open R itself, you're confronted with a few disclaimers and a command prompt, similar in appearance to the Terminal on macOS or command prompt in Windows.

\includegraphics[width=7.99in]{atlas/r_screenshot}

This looks a bit intimidating, but you'll almost never open R directly and interact with it in that way.

To analyse data with R, you will typically write out a text file containing your code. This file - which we'll call a script - should be able to be read and executed by R from start to finish. The easiest way to write your code, run your script, and generate your outputs (whether that's a chart, a document, or a set of model results) is to use RStudio.

\hypertarget{what-is-rstudio}{%
\section{What is RStudio?}\label{what-is-rstudio}}

RStudio is another piece of free software you can download and run on your computer.\footnote{RStudio is, somewhat confusingly, a product made by a company called RStudio. Although the RStudio desktop software is free, RStudio makes money by charging for other services, like running R in the cloud. When we refer to RStudio, we're referring to the desktop software unless we make it clear that we mean the company.} It's also available for Windows, macOS and Linux. In programmer jargon, RStudio is an ``integrated development environment'' or IDE. This means RStudio has a range of tools that help you work with R. It has a text editor for you to write R scripts, an R `console' to interact directly with the language, and panes that let you see the objects you have stored in memory and any graphs you've created.

\includegraphics[width=18.4in]{atlas/rstudio_screenshot}

You'll almost always interact with R by opening RStudio.

\hypertarget{installing-r-and-rstudio}{%
\section{Installing R and RStudio}\label{installing-r-and-rstudio}}

Although you'll usually work with R by opening RStudio, you need to install both R and RStudio separately.

Install R by going to \href{https://cran.r-project.org}{CRAN}, the Comprehensive R Archive Network. CRAN is a community-run website that houses R itself as well as a broad range of R packages.

\includegraphics[width=15.69in]{atlas/r_cran}

You want to download the latest base R release, as a `binary'. Don't worry, you don't need to know what a binary is.

For macOS, the page will look like this:

\includegraphics[width=15.68in]{atlas/r_cran_macos}

For Windows, you'll need to click on the `base' version, and then click again to start the download.

\includegraphics[width=15.69in]{atlas/r_cran_windows_1}
\includegraphics[width=15.67in]{atlas/r_cran_windows_2}

Once you've installed R, you'll need to install RStudio. Go to the \href{https://www.rstudio.com/products/rstudio/download/\#download}{RStudio website and install the latest version} of RStudio Desktop (open source license).

Once they're both installed, get started by opening RStudio.

\hypertarget{packages}{%
\section{Packages}\label{packages}}

R comes with a lot of functions - commands - built in to do a broad range of data tasks. You could, if you really wanted, import a dataset, clean it up, estimate a model, and make a plot all using the functions that come with R - known as `base R'\footnote{Technically some of the `built-in' functions are part of packages, like the \texttt{tools}, \texttt{utils} and \texttt{stats} packages that come with R. We'll refer to all these as base R.}.

But a lot of our work at Grattan uses add-on software to base R, known as `packages'. Some packages, like the popular `dplyr', make it quicker and/or easier to do tasks that you could otherwise do in base R. Other packages expand the possibilities of what R can do - like fitting a machine learning model, for example.

Like R itself, packages are free and open source. You can install them from within RStudio.

At Grattan, we make heavy use of a set of related packages known collectively as the \texttt{tidyverse}. We'll cover these more in a later chapter.

\hypertarget{installing-packages}{%
\subsection{Installing packages}\label{installing-packages}}

You'll typically install packages using the console in RStudio. That's the part of the window that, by default, sits in the bottom-left corner of the screen.

In our work at Grattan, we use packages from two different source: CRAN and Github. The main difference you need to know about is that we use different commands to install packages from these two sources.

To install a package from CRAN, we use the command \texttt{install.packages()}.

For example, this code will install the \texttt{ggplot2} package from CRAN:

\begin{Shaded}
\begin{Highlighting}[]
\KeywordTok{install.packages}\NormalTok{(}\StringTok{"ggplot2"}\NormalTok{)}
\end{Highlighting}
\end{Shaded}

To install a package from Github, we use the function \texttt{install\_github()}. Unfortunately, this package doesn't come with R - it's part of the \texttt{devtools} package. First, we install devtools from CRAN:

\begin{Shaded}
\begin{Highlighting}[]
\KeywordTok{install.packages}\NormalTok{(}\StringTok{"devtools"}\NormalTok{)}
\end{Highlighting}
\end{Shaded}

Now we can install packages from Github using the \texttt{install\_github()} function from the \texttt{devtools} package. For example, here's how we would install the Grattan ggplot2 theme, which we'll discuss later in this website:

\begin{Shaded}
\begin{Highlighting}[]
\NormalTok{devtools}\OperatorTok{::}\KeywordTok{install_github}\NormalTok{(}\StringTok{"mattcowgill/grattantheme"}\NormalTok{, }\DataTypeTok{dependencies =} \OtherTok{TRUE}\NormalTok{)}
\end{Highlighting}
\end{Shaded}

\hypertarget{using-packages}{%
\subsection{Using packages}\label{using-packages}}

Before using a function that comes from a package, as opposed to base R, you need to tell R where to look for the function. There are two main ways to do that.

We can either load (aka `attach') the package by using the \texttt{library()} function:

\begin{Shaded}
\begin{Highlighting}[]
\KeywordTok{library}\NormalTok{(devtools)}

\CommentTok{# Now that the `devtools` package is loaded, we can use its `install_github()` function:}

\KeywordTok{install_github}\NormalTok{(}\StringTok{"mattcowgill/grattantheme"}\NormalTok{)}
\end{Highlighting}
\end{Shaded}

Or, we can use two colons - \texttt{::} - to tell R to use an individual function from a package without loading it:

\begin{Shaded}
\begin{Highlighting}[]
\NormalTok{devtools}\OperatorTok{::}\KeywordTok{install_github}\NormalTok{(}\StringTok{"mattcowgill/grattantheme"}\NormalTok{)}
\end{Highlighting}
\end{Shaded}

It usually makes sense to load a package with \texttt{library()}, unless you only need to use one of its function once or twice. There's no harm to using the \texttt{::} operator even if you have already loaded a package with \texttt{library()}. This can remove ambiguity both for R and for humans reading your code, particularly if you're using an obscure function - it makes it clearer where the function comes from.

\hypertarget{why-use-r}{%
\chapter{Why use R?}\label{why-use-r}}

We can break this question into two parts:
1. Why use script-based software to analyse data?
2. Why use R, specifically?

\hypertarget{why-script}{%
\section{Why use script-based software?}\label{why-script}}

\begin{enumerate}
\def\labelenumi{\arabic{enumi}.}
\tightlist
\item
  Make your analysis reproducible by setting out the complete series of steps taken from raw data to final output.
\item
  Work with large data sets.
\end{enumerate}

\hypertarget{why-R}{%
\section{Why use R specifically?}\label{why-R}}

\begin{Shaded}
\begin{Highlighting}[]
\KeywordTok{library}\NormalTok{(tidyverse)}
\end{Highlighting}
\end{Shaded}

\hypertarget{intro}{%
\chapter{Using R at Grattan}\label{intro}}

\hypertarget{using-r-projects-for-a-fully-reproducible-workflow.}{%
\section{Using R projects for a fully reproducible workflow.}\label{using-r-projects-for-a-fully-reproducible-workflow.}}

\emph{Finally adhering to the `hit by a bus' rule.}

Having a clear, consistent structure for our analyses means that our work is more easily checked and revised, including by ourselves in the future. A small investment of time up front to set up your analysis will save time (your own and others') down the track.

Cover:
1. setwd() and machine-speficic filepaths are bad
2. relative file paths are good
3. RStudio projects are an easy, reproducible way to set your wd

\hypertarget{filepaths}{%
\subsection{Filepaths}\label{filepaths}}

Filepaths should be relative to the working directory, and the working directory should be set by the project.

\textbf{Good}

\begin{Shaded}
\begin{Highlighting}[]
\NormalTok{hes <-}\StringTok{ }\KeywordTok{read_csv}\NormalTok{(}\StringTok{"data/HES/hes1516.csv"}\NormalTok{)}
\KeywordTok{grattan_save}\NormalTok{(}\StringTok{"images/expenditure_by_income.pdf"}\NormalTok{)}
\end{Highlighting}
\end{Shaded}

\textbf{Bad}

\begin{Shaded}
\begin{Highlighting}[]
\NormalTok{hes <-}\StringTok{ }\KeywordTok{read_csv}\NormalTok{(}\StringTok{"/Users/mcowgill/Desktop/hes1516.csv"}\NormalTok{)}
\NormalTok{hes <-}\StringTok{ }\KeywordTok{read_csb}\NormalTok{(}\StringTok{"C:\textbackslash{}Users\textbackslash{}mcowgill\textbackslash{}Desktop\textbackslash{}hes1516.csv"}\NormalTok{)}
\KeywordTok{grattan_save}\NormalTok{(}\StringTok{"/Users/mcowgill/Desktop/images/expenditure_by_income.pdf"}\NormalTok{)}
\end{Highlighting}
\end{Shaded}

\hypertarget{keep-your-scripts-manageable}{%
\subsection{Keep your scripts manageable}\label{keep-your-scripts-manageable}}

As a general rule of thumb, use one script per output. It should be clear what your script is trying to do (use comments!).

Consider breaking your analysis into pieces. For example:

\begin{itemize}
\tightlist
\item
  01\_import.R
\item
  02\_tidy.R
\item
  03\_model.R
\item
  04\_visualise.R
\end{itemize}

\textbf{Don't} include interactive work (like \texttt{View(mydf)}, \texttt{str(mydf)}, \texttt{mean(mydf\$variable)}, etc.) in your saved script.

\hypertarget{use-subfolders-of-your-project-folder}{%
\subsection{Use subfolders of your project folder}\label{use-subfolders-of-your-project-folder}}

Remember the hit-by-a-bus rule. It should be easy for any Grattan colleague to open your project folder and get up to speed with what it does. Putting all your files - raw data, scripts, output - in the one folder makes it harder to understand how your work fits together.

Use subfolders to clearly separate your code, raw data, and output.

\hypertarget{grattan-coding-style-guide}{%
\section{Grattan coding style guide}\label{grattan-coding-style-guide}}

Short summary of why

Link to style guide

\hypertarget{what-is-the-tidyverse-and-why-do-we-use-it}{%
\section{What is the tidyverse and why do we use it?}\label{what-is-the-tidyverse-and-why-do-we-use-it}}

Introduce following chapters

\hypertarget{an-introduction-to-rmarkdown}{%
\section{An introduction to RMarkdown}\label{an-introduction-to-rmarkdown}}

\hypertarget{resources-in-this-package}{%
\section{Resources in this package}\label{resources-in-this-package}}

\begin{itemize}
\tightlist
\item
  Starting a piece of analysis `cheat sheet'.
\item
  Updated style guide.
\item
  Written guide/slides.
\end{itemize}

\hypertarget{data-visualisation}{%
\chapter{Data Visualisation}\label{data-visualisation}}

This chapter explores

\hypertarget{introduction-to-data-visualisation}{%
\section{Introduction to data visualisation}\label{introduction-to-data-visualisation}}

Data visualisation is used in two broad ways:

\begin{enumerate}
\def\labelenumi{\arabic{enumi}.}
\tightlist
\item
  to examine and explore your data; and
\item
  to present a finding to your audience.
\end{enumerate}

When you start using a dataset, you should \emph{look at it}.\footnote{From Kieran Healy's \emph{Data Vizualization: A Practical Introduction)} (\href{https://socviz.co/}{available free}): `You should look at your data. Graphs and charts let you explore and learn about the structure of the information you collect. Good data visualizations also make it easier to communicate your ideas and findings to other people.'} Plot histograms of variables-of-interest to spot outliers. Explore correlations with scatter plots and lines-of-best-fit. Check how many observations are in particular groups with bar charts. Identify variables that have missing or coded-missing values. Use faceting to explore differences in the above between groups, and do it interactively with non-static plots.

These \textbf{exploratory plots} are just for you and your team. They don't need to be perfectly labelled, the right size, in the Grattan palette or be particularly interesting.
They're built and used to explore the data.
Through this process, you can become confident your data is \emph{what it says it is}.

When you \textbf{present a visualisation to a reader}, you make decisions about what they can and cannot see. You choose to highlight or omit particular things to help them better understand the message you are presenting.

This requires important technical decisions: what data to use, what `stat' to present it with --- show every data point, show a distribution function, show the average or the median --- and on what scale --- raw numbers, on a log scale, as a proportion of a total.

It also requires \emph{aesthetic} decisions. What colours in the Grattan palette would work best? Where should the labels be placed and how could they be phrased to succinctly convey meaning? Should data points be represented by lines, or bars, or dots, or balloons, or shades of colour?

All of these decisions need to made with two things in mind:

\begin{enumerate}
\def\labelenumi{\arabic{enumi}.}
\tightlist
\item
  Rigour, accuracy, legitimacy: the chart needs to be honest.
\item
  The reader: the chart needs to help the reader understand something, and it must convince them to pay attention.
\end{enumerate}

At the margins, sometimes these two ideas can be in conflict: maybe a 70-word definition in the middle of your chart would improve its technical accuracy, but it could confuse the average reader.

Similarly, a bar chart is often the safest way to display data. But if the reader has stopped paying attention by your sixth consecutive bar chart, your point loses its punch.\footnote{`Bar charts are evidence that you are dead inside' -- Amanda Cox, data editor for the New York Times.}

The way we design charts -- much like our writing -- should always be honest, clear and engaging to the reader.

This chapter shows how you can do this with R. It starts with the `grammar of graphics' concepts of a package called \texttt{ggplot}, explains how to make those charts `Grattan-y', then provides examples for all common (and some not-so-common) charts you should add to your box of data visualisation tools to impress your message on our readers.

\hypertarget{set-up-and-packages}{%
\section{Set-up and packages}\label{set-up-and-packages}}

This section uses the package \texttt{ggplot2} to visualise data, and \texttt{dplyr} functions to manipulate data. Both of these packages are loaded with \texttt{tidyverse}. The \texttt{scales} package helps with labelling your axes.

The \texttt{grattantheme} package is used to make charts look Grattan-y. The \texttt{absmapsdata} package is used to help make maps.

\begin{Shaded}
\begin{Highlighting}[]
\KeywordTok{library}\NormalTok{(tidyverse)}
\KeywordTok{library}\NormalTok{(grattantheme)}
\KeywordTok{library}\NormalTok{(ggrepel)}
\KeywordTok{library}\NormalTok{(absmapsdata)}
\KeywordTok{library}\NormalTok{(sf)}
\KeywordTok{library}\NormalTok{(scales)}
\end{Highlighting}
\end{Shaded}

For most charts in this chapter, we'll use the \texttt{sa3\_income} data summarised below\footnote{From (ABS Employee income by occupation and sex, 2010-11 to 2015-16){[}\url{https://www.abs.gov.au/AUSSTATS/abs@.nsf/DetailsPage/6524.0.55.0022011-2016?OpenDocument}{]}} It is a long dataset containing the median income and number of workers by SA3, occupation and sex between 2010 and 2015. We will also create a \texttt{professionals} subset that only includes people in professional occupations in 2015:

\begin{Shaded}
\begin{Highlighting}[]
\NormalTok{sa3_income <-}\StringTok{ }\KeywordTok{read_csv}\NormalTok{(}\StringTok{"data/sa3_income.csv"}\NormalTok{) }\OperatorTok\StringTok{ }
\StringTok{  }\KeywordTok{filter}\NormalTok{(}\OperatorTok{!}\KeywordTok{is.na}\NormalTok{(median_income),}
\NormalTok{         sex }\OperatorTok{!=}\StringTok{ "Persons"}\NormalTok{)}

\NormalTok{professionals <-}\StringTok{ }\NormalTok{sa3_income }\OperatorTok\StringTok{ }
\StringTok{  }\KeywordTok{filter}\NormalTok{(year }\OperatorTok{==}\StringTok{ }\DecValTok{2015}\NormalTok{,}
\NormalTok{         occupation }\OperatorTok{==}\StringTok{ "Professionals"}\NormalTok{) }

\CommentTok{# Show the first six rows of the new dataset}
\KeywordTok{head}\NormalTok{(sa3_income)}
\end{Highlighting}
\end{Shaded}

\begin{verbatim}
## # A tibble: 6 x 10
##     sa3 sa3_name sa3_sqkm sa4_name gcc_name occupation sex    year
##   <dbl> <chr>       <dbl> <chr>    <chr>    <chr>      <chr> <dbl>
## 1 10102 Queanbe~    6511. Capital~ Rest of~ Clerical ~ Fema~  2010
## 2 10102 Queanbe~    6511. Capital~ Rest of~ Clerical ~ Fema~  2011
## 3 10102 Queanbe~    6511. Capital~ Rest of~ Clerical ~ Fema~  2012
## 4 10102 Queanbe~    6511. Capital~ Rest of~ Clerical ~ Fema~  2013
## 5 10102 Queanbe~    6511. Capital~ Rest of~ Clerical ~ Fema~  2014
## 6 10102 Queanbe~    6511. Capital~ Rest of~ Clerical ~ Fema~  2015
## # ... with 2 more variables: median_income <dbl>, persons <dbl>
\end{verbatim}

\hypertarget{concepts}{%
\section{Concepts}\label{concepts}}

The \texttt{ggplot2} package is based on the \textbf{g}rammar of \textbf{g}raphics. \ldots{}

The main ingredients to a \texttt{ggplot} chart are:

\begin{itemize}
\tightlist
\item
  \textbf{Data}: what data should be plotted.

  \begin{itemize}
  \tightlist
  \item
    e.g. \texttt{data}
  \end{itemize}
\item
  \textbf{Aesthetics}: what variables should be linked to what chart elements.

  \begin{itemize}
  \tightlist
  \item
    e.g. \texttt{aes(x\ =\ population,\ y\ =\ age)} to connect the \texttt{population} variable to the \texttt{x} axis, and the \texttt{age} variable to the \texttt{y} axis.
  \end{itemize}
\item
  \textbf{Geoms}: how the data should be plotted.

  \begin{itemize}
  \tightlist
  \item
    e.g. \texttt{geom\_point()} will produce a scatter plot, \texttt{geom\_col} will produce a column chart, \texttt{geom\_line()} will produce a line chart.
  \end{itemize}
\end{itemize}

Each plot you make will be made up of these three elements. The \href{https://ggplot2.tidyverse.org/reference/}{full list of standard geoms} is listed in the \texttt{tidyverse} documentation.

\texttt{ggplot} also has a `cheat sheet' that contains many of the often-used elements of a plot, which you can download \href{https://github.com/rstudio/cheatsheets/raw/master/data-visualization-2.1.pdf}{here}.

\begin{center}\includegraphics[width=17.08in]{atlas/ggplot_cheat_sheet} \end{center}

For example, you can plot a column chart by passing the \texttt{sa3\_income} dataset into \texttt{ggplot()} (``make a chart with this data''). This completes the first step -- data -- and produces an empty plot:

\begin{Shaded}
\begin{Highlighting}[]
\NormalTok{professionals }\OperatorTok\StringTok{ }
\StringTok{        }\KeywordTok{ggplot}\NormalTok{()}
\end{Highlighting}
\end{Shaded}

\includegraphics{Data_visualisation_files/figure-latex/empty_plot-1.pdf}

Next, set the \texttt{aes} (aesthetics) to \texttt{x\ =\ state} (``make the x-axis represent state''), \texttt{y\ =\ pop} (``the y-axis should represent population''), and \texttt{fill\ =\ year} (``the fill colour represents year''). Now \texttt{ggplot} knows where things should \emph{go}.

If we just plot that, you'll see that \texttt{ggplot} knows a little bit more about what we're trying to do. It has the states on the x-axis and range of populations on the y-axis:

\begin{Shaded}
\begin{Highlighting}[]
\NormalTok{professionals }\OperatorTok\StringTok{ }
\StringTok{        }\KeywordTok{ggplot}\NormalTok{(}\KeywordTok{aes}\NormalTok{(}\DataTypeTok{x =}\NormalTok{ persons,}
                   \DataTypeTok{y =}\NormalTok{ median_income,}
                   \DataTypeTok{colour =}\NormalTok{ sex))}
\end{Highlighting}
\end{Shaded}

\includegraphics{Data_visualisation_files/figure-latex/empty_aes-1.pdf}

Now that \texttt{ggplot} knows where things should go, it needs to how to \emph{plot} them on the chart. For this we use \texttt{geoms}. Tell \texttt{ggplot} to take the things it knows and plot them as a column chart by using \texttt{geom\_col}:

\begin{Shaded}
\begin{Highlighting}[]
\NormalTok{professionals }\OperatorTok
\StringTok{        }\KeywordTok{ggplot}\NormalTok{(}\KeywordTok{aes}\NormalTok{(}\DataTypeTok{x =}\NormalTok{ persons,}
                   \DataTypeTok{y =}\NormalTok{ median_income,}
                   \DataTypeTok{colour =}\NormalTok{ sex)) }\OperatorTok{+}\StringTok{ }
\StringTok{        }\KeywordTok{geom_point}\NormalTok{()}
\end{Highlighting}
\end{Shaded}

\includegraphics{Data_visualisation_files/figure-latex/complete_plot-1.pdf}

Great! There are a couple of quick things we can do to make the chart a bit clearer. There are points for each group in each year, which we probably don't need. So filter the data before you pass it to \texttt{ggplot} to just include 2015: \texttt{filter(year\ ==\ 2015)}. There will still be lots of overlapping points, so set the opacity to below one with \texttt{alpha\ =\ 0.5}. The \texttt{persons} x-axis can be changed to a log scale with \texttt{scale\_x\_log10}.

\begin{Shaded}
\begin{Highlighting}[]
\NormalTok{professionals }\OperatorTok\StringTok{ }
\StringTok{        }\KeywordTok{ggplot}\NormalTok{(}\KeywordTok{aes}\NormalTok{(}\DataTypeTok{x =}\NormalTok{ persons,}
                   \DataTypeTok{y =}\NormalTok{ median_income,}
                   \DataTypeTok{colour =}\NormalTok{ sex)) }\OperatorTok{+}\StringTok{ }
\StringTok{        }\KeywordTok{geom_point}\NormalTok{(}\DataTypeTok{alpha =} \FloatTok{.5}\NormalTok{) }\OperatorTok{+}\StringTok{ }
\StringTok{        }\KeywordTok{scale_x_log10}\NormalTok{()}
\end{Highlighting}
\end{Shaded}

\includegraphics{Data_visualisation_files/figure-latex/with_changes-1.pdf}

That looks a bit better. The following sections in this chapter will cover a broad range of charts and designs, but they will all use the same building-blocks of \texttt{data}, \texttt{aes}, and \texttt{geom}.

The rest of the chapter will explore:

\begin{itemize}
\tightlist
\item
  Exploratory data visualisation
\item
  Grattanising your charts and choosing colours
\item
  Saving charts according to Grattan templates
\item
  Making bar, line, scatter and distribution plots
\item
  Making maps and interactive charts
\item
  Adding chart labels
\end{itemize}

\hypertarget{exploratory-data-visualisation}{%
\section{Exploratory data visualisation}\label{exploratory-data-visualisation}}

Plotting your data early in the analysis stage can help you quickly identify outliers, oddities, things that don't look quite right.

\hypertarget{making-grattan-y-charts}{%
\section{Making Grattan-y charts}\label{making-grattan-y-charts}}

The \texttt{grattantheme} package contains functions that help \emph{Grattanise} your charts. It is hosted here: \url{https://github.com/mattcowgill/grattantheme}

You can install it with \texttt{remotes::install\_github} from the package:

\begin{Shaded}
\begin{Highlighting}[]
\KeywordTok{install.packages}\NormalTok{(}\StringTok{"remotes"}\NormalTok{)}
\NormalTok{remotes}\OperatorTok{::}\KeywordTok{install_github}\NormalTok{(}\StringTok{"mattcowgill/grattantheme"}\NormalTok{)}
\end{Highlighting}
\end{Shaded}

The key functions of \texttt{grattantheme} are:

\begin{itemize}
\tightlist
\item
  \texttt{theme\_grattan}: set size, font and colour defaults that adhere to the Grattan style guide.
\item
  \texttt{grattan\_y\_continuous}: sets the right defaults for a continuous y-axis.
\item
  \texttt{grattan\_colour\_continuous}: pulls colours from the Grattan colour palette for \texttt{colour} aesthetics.
\item
  \texttt{grattan\_fill\_continuous}: pulls colours from the Grattan colour palette for \texttt{fill} aesthetics.
\item
  \texttt{grattan\_save}: a save function that exports charts in correct report or presentation dimensions.
\end{itemize}

This section will run through some examples of \emph{Grattanising} charts. The \texttt{ggplot} functions are explored in more detail in the next section.

\hypertarget{making-grattan-charts}{%
\subsection{Making Grattan charts}\label{making-grattan-charts}}

Start with a scatterplot, similar to the one made above:

\begin{Shaded}
\begin{Highlighting}[]
\NormalTok{base_chart <-}\StringTok{ }\NormalTok{professionals }\OperatorTok\StringTok{ }
\StringTok{        }\KeywordTok{ggplot}\NormalTok{(}\KeywordTok{aes}\NormalTok{(}\DataTypeTok{x =}\NormalTok{ persons,}
                   \DataTypeTok{y =}\NormalTok{ median_income,}
                   \DataTypeTok{colour =}\NormalTok{ sex)) }\OperatorTok{+}\StringTok{ }
\StringTok{        }\KeywordTok{geom_point}\NormalTok{(}\DataTypeTok{alpha =} \FloatTok{.5}\NormalTok{) }\OperatorTok{+}\StringTok{ }
\StringTok{        }\KeywordTok{labs}\NormalTok{(}\DataTypeTok{title =} \StringTok{"More professionals, the more they earn"}\NormalTok{,}
             \DataTypeTok{subtitle =} \StringTok{"SA3 areas by number of professionals and thier median income"}\NormalTok{,}
             \DataTypeTok{x =} \StringTok{"Number of professionals"}\NormalTok{,}
             \DataTypeTok{y =} \StringTok{"Median income"}\NormalTok{,}
             \DataTypeTok{caption =} \StringTok{"Source: ABS Estimates of Personal Income for Small Areas, 2011-2016"}\NormalTok{)}

\NormalTok{base_chart}
\end{Highlighting}
\end{Shaded}

\includegraphics{Data_visualisation_files/figure-latex/base_chart-1.pdf}

Let's make it Grattany. First, add \texttt{theme\_grattan} to your plot:

\begin{Shaded}
\begin{Highlighting}[]
\NormalTok{base_chart }\OperatorTok{+}
\StringTok{        }\KeywordTok{theme_grattan}\NormalTok{()}
\end{Highlighting}
\end{Shaded}

\includegraphics{Data_visualisation_files/figure-latex/add_theme_grattan-1.pdf}

Then \texttt{grattan\_y\_continuous} to align the x-axis with zero. This function takes the same arguments as \texttt{scale\_y\_continuous}, so you can add \texttt{labels\ =\ comma()} to reformat the y-axis labels:

\begin{Shaded}
\begin{Highlighting}[]
\NormalTok{base_chart }\OperatorTok{+}
\StringTok{        }\KeywordTok{theme_grattan}\NormalTok{() }\OperatorTok{+}
\StringTok{        }\KeywordTok{grattan_y_continuous}\NormalTok{(}\DataTypeTok{labels =}\NormalTok{ dollar) }\OperatorTok{+}
\StringTok{        }\KeywordTok{scale_x_log10}\NormalTok{(}\DataTypeTok{labels =}\NormalTok{ comma) }
\end{Highlighting}
\end{Shaded}

\includegraphics{Data_visualisation_files/figure-latex/add_grattan_y_continuous-1.pdf}

To define \texttt{colour} colours, use \texttt{grattan\_colour\_manual} with the number of colours you need (two, in this case):

\begin{Shaded}
\begin{Highlighting}[]
\NormalTok{prof_chart <-}\StringTok{ }\NormalTok{base_chart }\OperatorTok{+}
\StringTok{        }\KeywordTok{theme_grattan}\NormalTok{() }\OperatorTok{+}
\StringTok{        }\KeywordTok{grattan_y_continuous}\NormalTok{(}\DataTypeTok{labels =}\NormalTok{ dollar) }\OperatorTok{+}
\StringTok{        }\KeywordTok{scale_x_log10}\NormalTok{(}\DataTypeTok{labels =}\NormalTok{ comma) }\OperatorTok{+}
\StringTok{        }\KeywordTok{grattan_colour_manual}\NormalTok{(}\DecValTok{2}\NormalTok{) }

\NormalTok{prof_chart}
\end{Highlighting}
\end{Shaded}

\includegraphics{Data_visualisation_files/figure-latex/add_fill-1.pdf}

Nice chart! Now you can save it and share it with the world.

\hypertarget{saving-grattan-charts}{%
\subsection{Saving Grattan charts}\label{saving-grattan-charts}}

The \texttt{grattan\_save} function saves your charts according to Grattan templates. It takes these arguments:

\begin{itemize}
\tightlist
\item
  \texttt{filename}: the path, name and file-type of your saved chart. eg: \texttt{"atlas/professionals\_chart.pdf"}.
\item
  \texttt{object}: the R object that you want to save. eg: \texttt{prof\_chart}. If left blank, it grabs the last chart that was displayed.
\item
  \texttt{type}: the Grattan template to be used. This is one of:

  \begin{itemize}
  \tightlist
  \item
    \texttt{"normal"} The default. Use for normal Grattan report charts, or to paste into a 4:3 PowerPoint slide. Width: 22.2cm, height: 14.5cm.
  \item
    \texttt{"normal\_169"} Only useful for pasting into a 16:9 format Grattan PowerPoint slide. Width: 30cm, height: 14.5cm.
  \item
    \texttt{"tiny"} Fills the width of a column in a Grattan report, but is shorter than usual. Width: 22.2cm, height: 11.1cm.
  \item
    \texttt{"wholecolumn"} Takes up a whole column in a Grattan report. Width: 22.2cm, height: 22.2cm.
  \item
    \texttt{"fullpage"} Fills a whole page of a Grattan report. Width: 44.3cm, height: 22.2cm.
  \item
    \texttt{"fullslide"} Creates an image that looks like a 4:3 Grattan PowerPoint slide, complete with logo. Width: 25.4cm, height: 19.0cm.
  \item
    \texttt{"fullslide\_169"} Creates` an image that looks like a 16:9 Grattan PowerPoint slide, complete with logo. Use this to drop into standard presentations. Width: 33.9cm, height: 19.0cm
  \item
    \texttt{"blog"} Creates a 4:3 image that looks like a Grattan PowerPoint slide, but with less border whitespace than `fullslide'."
  \item
    \texttt{"fullslide\_44"\ Creates} an image that looks like a 4:4 Grattan PowerPoint slide. This may be useful for taller charts for the Grattan blog; not useful for any other purpose. Width: 25.4cm, height: 25.4cm.
  \item
    Set \texttt{type\ =\ "all"} to save your chart in all available sizes.
  \end{itemize}
\item
  \texttt{height}: override the height set by \texttt{type}. This can be useful for really long charts in blogposts.
\item
  \texttt{save\_data}: exports a \texttt{csv} file containing the data used in the chart.
\item
  \texttt{force\_labs}: override the removal of labels for a particular \texttt{type}. eg \texttt{force\_labs\ =\ TRUE} will keep the y-axis label.
\end{itemize}

To save the \texttt{prof\_chart} plot created above as a whole-column chart for a \textbf{report}:

\begin{Shaded}
\begin{Highlighting}[]
\KeywordTok{grattan_save}\NormalTok{(}\StringTok{"atlas/professionals_chart_report.pdf"}\NormalTok{, prof_chart, }\DataTypeTok{type =} \StringTok{"wholecolumn"}\NormalTok{)}
\end{Highlighting}
\end{Shaded}

\includegraphics[width=38.76in]{atlas/professionals_chart_report}

To save it as a \textbf{presentation} slide instead, use \texttt{type\ =\ "fullslide"}:

\begin{Shaded}
\begin{Highlighting}[]
\KeywordTok{grattan_save}\NormalTok{(}\StringTok{"atlas/professionals_chart_presentation.pdf"}\NormalTok{, prof_chart, }\DataTypeTok{type =} \StringTok{"fullslide"}\NormalTok{)}
\end{Highlighting}
\end{Shaded}

\includegraphics[width=44.44in]{atlas/professionals_chart_presentation}

Or, if you want to emphasise the point in a \emph{really tall} chart for a \textbf{blogpost}, you can use \texttt{type\ =\ "blog"} and adjust the \texttt{height} to be 50cm. Also note that because this is for the blog, you should save it as a \texttt{png} file:

\begin{Shaded}
\begin{Highlighting}[]
\KeywordTok{grattan_save}\NormalTok{(}\StringTok{"atlas/professionals_chart_blog.png"}\NormalTok{, prof_chart, }
             \DataTypeTok{type =} \StringTok{"blog"}\NormalTok{, }\DataTypeTok{height =} \DecValTok{30}\NormalTok{)}
\end{Highlighting}
\end{Shaded}

\includegraphics[width=44.44in]{atlas/professionals_chart_blog}

And that's it! The following sections will go into more detail about different chart types in R, but you'll mostly use the same basic \texttt{grattantheme} formatting you've used here.

\hypertarget{adding-labels}{%
\section{Adding labels}\label{adding-labels}}

Labels can be a bit finicky -- especially compared to labelling charts visually in PowerPoint. \ldots{}

Labels can be done in two broad ways:

\begin{enumerate}
\def\labelenumi{\arabic{enumi}.}
\tightlist
\item
  Labelling every single data point on your chart. Grattan charts rarely do this.
\item
  Labelling some of the data points on your chart. This is how you label Grattan charts: label on item in a group and let the reader join the dots.
\end{enumerate}

We'll look at the first approach so you can get a feel for how the labelling geoms -- \texttt{geom\_label} and \texttt{geom\_text} (and some useful extensions) -- work. It won't be pretty.

\begin{Shaded}
\begin{Highlighting}[]
\NormalTok{prof_chart }\OperatorTok{+}
\StringTok{  }\KeywordTok{geom_text}\NormalTok{(}\KeywordTok{aes}\NormalTok{(}\DataTypeTok{label =}\NormalTok{ sex))}
\end{Highlighting}
\end{Shaded}

\includegraphics{Data_visualisation_files/figure-latex/add_annotate-1.pdf}

Great! That looks \emph{terrible}. \texttt{geom\_text} is labelling each individual point because it has been told to do so. Just like \texttt{geom\_point}, it takes the \texttt{x} and \texttt{y} aesthetics of each observation, then plots the \texttt{label} at that location. But we just want to label one of the points for \texttt{female} and one for \texttt{male}.

To do this, we can create a new dataset that just contains one observation each. Here, you're filtering the dataset to include \emph{only} the female/male observations that have the most people:

\begin{Shaded}
\begin{Highlighting}[]
\NormalTok{label_data <-}\StringTok{ }\NormalTok{professionals }\OperatorTok\StringTok{ }
\StringTok{  }\KeywordTok{group_by}\NormalTok{(sex) }\OperatorTok\StringTok{ }
\StringTok{  }\KeywordTok{filter}\NormalTok{(persons }\OperatorTok{==}\StringTok{ }\KeywordTok{max}\NormalTok{(persons))}

\NormalTok{label_data}
\end{Highlighting}
\end{Shaded}

\begin{verbatim}
## # A tibble: 2 x 10
## # Groups:   sex [2]
##     sa3 sa3_name sa3_sqkm sa4_name gcc_name occupation sex    year
##   <dbl> <chr>       <dbl> <chr>    <chr>    <chr>      <chr> <dbl>
## 1 11703 Sydney ~     25.1 Sydney ~ Greater~ Professio~ Fema~  2015
## 2 11703 Sydney ~     25.1 Sydney ~ Greater~ Professio~ Males  2015
## # ... with 2 more variables: median_income <dbl>, persons <dbl>
\end{verbatim}

And then tell \texttt{geom\_text} to look at \emph{that} dataset:

\begin{Shaded}
\begin{Highlighting}[]
\NormalTok{prof_chart }\OperatorTok{+}
\StringTok{  }\KeywordTok{geom_text}\NormalTok{(}\DataTypeTok{data =}\NormalTok{ label_data,}
            \KeywordTok{aes}\NormalTok{(}\DataTypeTok{label =}\NormalTok{ sex))}
\end{Highlighting}
\end{Shaded}

\includegraphics{Data_visualisation_files/figure-latex/unnamed-chunk-2-1.pdf}

\hypertarget{reading-data}{%
\chapter{Reading data}\label{reading-data}}

\hypertarget{importing-data}{%
\section{Importing data}\label{importing-data}}

\hypertarget{reading-csv-files}{%
\subsection{Reading CSV files}\label{reading-csv-files}}

\hypertarget{read_csv}{%
\subsubsection{\texorpdfstring{\texttt{read\_csv()}}{read\_csv()}}\label{read_csv}}

The \texttt{read\_csv()} function from the \texttt{tidyverse} is quicker and smarter than \texttt{read.csv} in base R.

Pitfalls:
1. read\_csv is quicker because it surveys a sample of the data

We can also compress \texttt{.csv} files into \texttt{.zip} files and read them \emph{directly} using \texttt{read\_csv()}:

\begin{Shaded}
\begin{Highlighting}[]
\KeywordTok{read_csv}\NormalTok{(}\StringTok{"data/my_data.zip"}\NormalTok{)}
\end{Highlighting}
\end{Shaded}

This is useful for two reasons:

\begin{enumerate}
\def\labelenumi{\arabic{enumi}.}
\tightlist
\item
  The data takes up less room on your computer; and
\item
  The original data, which shouldn't ever be directly edited, is protected and cannot be directly edited.
\end{enumerate}

\hypertarget{data.tablefread}{%
\subsubsection{\texorpdfstring{\texttt{data.table::fread()}}{data.table::fread()}}\label{data.tablefread}}

The \texttt{fread} function from \texttt{data.table} is quicker than both \texttt{read.csv} and \texttt{read\_csv}.

\hypertarget{readxlread_excel}{%
\subsection{\texorpdfstring{\texttt{readxl::read\_excel()}}{readxl::read\_excel()}}\label{readxlread_excel}}

\hypertarget{rio}{%
\subsection{\texorpdfstring{\texttt{rio}}{rio}}\label{rio}}

\hypertarget{readabs}{%
\subsection{\texorpdfstring{\texttt{readabs}}{readabs}}\label{readabs}}

\hypertarget{reading-common-files}{%
\section{Reading common files:}\label{reading-common-files}}

\begin{itemize}
\tightlist
\item
  TableBuilder CSVSTRINGs
\item
  HES household file
\item
  SIH
\item
  LSAY and derivatives
\end{itemize}

See data directory for a list of microdata available to Grattan.

\hypertarget{appropriately-renaming-variables}{%
\section{Appropriately renaming variables}\label{appropriately-renaming-variables}}

As shown in the style guide

Add \texttt{rename\_abs} function to a common Grattan package?

\hypertarget{getting-to-tidy-data}{%
\section{Getting to tidy data}\label{getting-to-tidy-data}}

\texttt{pivot\_long()} and \texttt{pivot\_wide()}
\emph{Make sure these are stable btw}

\hypertarget{different-data-types}{%
\chapter{Different data types}\label{different-data-types}}

\hypertarget{tidy-data}{%
\section{Tidy data}\label{tidy-data}}

Other data structures

\hypertarget{dates-with-lubridate}{%
\section{\texorpdfstring{Dates with \texttt{lubridate::}}{Dates with lubridate::}}\label{dates-with-lubridate}}

The \texttt{lubridate::} package

\hypertarget{strings-with-stringr}{%
\section{\texorpdfstring{Strings with \texttt{stringr::}}{Strings with stringr::}}\label{strings-with-stringr}}

\begin{itemize}
\tightlist
\item
  Replacing values
\item
  Matching values
\item
  Separating columns
\end{itemize}

\hypertarget{factors-with-forcats}{%
\section{\texorpdfstring{Factors with \texttt{forcats::}}{Factors with forcats::}}\label{factors-with-forcats}}

\begin{itemize}
\tightlist
\item
  Dangers with factors
\end{itemize}

\hypertarget{data-transformation}{%
\chapter{Data transformation}\label{data-transformation}}

\hypertarget{the-pipe}{%
\section{The pipe}\label{the-pipe}}

\hypertarget{key-dplyr-functions}{%
\section{\texorpdfstring{Key \texttt{dplyr} functions:}{Key dplyr functions:}}\label{key-dplyr-functions}}

All have the same syntax structure, which enable pipe-chains.

\hypertarget{filter-with-filter}{%
\section{\texorpdfstring{Filter with \texttt{filter()}}{Filter with filter()}}\label{filter-with-filter}}

\hypertarget{arrange-with-arrange}{%
\section{\texorpdfstring{Arrange with \texttt{arrange()}}{Arrange with arrange()}}\label{arrange-with-arrange}}

\hypertarget{select-variables-with-select}{%
\section{\texorpdfstring{Select variables with \texttt{select()}}{Select variables with select()}}\label{select-variables-with-select}}

\hypertarget{group-data-with-group_by}{%
\section{\texorpdfstring{Group data with \texttt{group\_by()}}{Group data with group\_by()}}\label{group-data-with-group_by}}

\hypertarget{edit-and-add-new-variables-with-mutate}{%
\section{\texorpdfstring{Edit and add new variables with \texttt{mutate()}}{Edit and add new variables with mutate()}}\label{edit-and-add-new-variables-with-mutate}}

\hypertarget{cases-when-you-should-use-case_when}{%
\subsection{\texorpdfstring{Cases when you should use \texttt{case\_when()}}{Cases when you should use case\_when()}}\label{cases-when-you-should-use-case_when}}

\hypertarget{summarise-data-with-summarise}{%
\section{\texorpdfstring{Summarise data with \texttt{summarise()}}{Summarise data with summarise()}}\label{summarise-data-with-summarise}}

\hypertarget{joining-datasets-with-_join}{%
\section{\texorpdfstring{Joining datasets with \texttt{*\_join()}}{Joining datasets with *\_join()}}\label{joining-datasets-with-_join}}

\hypertarget{analysis}{%
\chapter{Analysis}\label{analysis}}

\hypertarget{creating-functions}{%
\chapter{Creating functions}\label{creating-functions}}

\hypertarget{it-can-be-useful-to-make-your-own-function}{%
\section{It can be useful to make your own function}\label{it-can-be-useful-to-make-your-own-function}}

Why on earth would you create your own function?

\hypertarget{defining-simple-functions}{%
\section{Defining simple functions}\label{defining-simple-functions}}

\hypertarget{more-complex-functions}{%
\section{More complex functions}\label{more-complex-functions}}

\hypertarget{sets-of-functions}{%
\section{Sets of functions}\label{sets-of-functions}}

\hypertarget{using-purrrmap}{%
\section{\texorpdfstring{Using \texttt{purrr::map}}{Using purrr::map}}\label{using-purrrmap}}

\hypertarget{sharing-your-useful-functions-with-grattan}{%
\section{Sharing your useful functions with Grattan}\label{sharing-your-useful-functions-with-grattan}}

\hypertarget{version-control}{%
\chapter{Version control}\label{version-control}}

\hypertarget{version-control-is-important-and-intimidating}{%
\section{Version control is important and intimidating}\label{version-control-is-important-and-intimidating}}

Version control is great!

\hypertarget{github}{%
\section{Github}\label{github}}

We use Github to version-control and share reports in LaTeX, so you're already a bit set-up.

\hypertarget{git}{%
\section{Git}\label{git}}

Using Git within R Studio\ldots{}

\bibliography{book.bib,packages.bib}


\end{document}
