\documentclass[]{book}
\usepackage{lmodern}
\usepackage{amssymb,amsmath}
\usepackage{ifxetex,ifluatex}
\usepackage{fixltx2e} % provides \textsubscript
\ifnum 0\ifxetex 1\fi\ifluatex 1\fi=0 % if pdftex
  \usepackage[T1]{fontenc}
  \usepackage[utf8]{inputenc}
\else % if luatex or xelatex
  \ifxetex
    \usepackage{mathspec}
  \else
    \usepackage{fontspec}
  \fi
  \defaultfontfeatures{Ligatures=TeX,Scale=MatchLowercase}
\fi
% use upquote if available, for straight quotes in verbatim environments
\IfFileExists{upquote.sty}{\usepackage{upquote}}{}
% use microtype if available
\IfFileExists{microtype.sty}{%
\usepackage{microtype}
\UseMicrotypeSet[protrusion]{basicmath} % disable protrusion for tt fonts
}{}
\usepackage{hyperref}
\hypersetup{unicode=true,
            pdftitle={Using R at Grattan Institute},
            pdfauthor={Will Mackey and Matt Cowgill},
            pdfborder={0 0 0},
            breaklinks=true}
\urlstyle{same}  % don't use monospace font for urls
\usepackage{natbib}
\bibliographystyle{apalike}
\usepackage{color}
\usepackage{fancyvrb}
\newcommand{\VerbBar}{|}
\newcommand{\VERB}{\Verb[commandchars=\\\{\}]}
\DefineVerbatimEnvironment{Highlighting}{Verbatim}{commandchars=\\\{\}}
% Add ',fontsize=\small' for more characters per line
\usepackage{framed}
\definecolor{shadecolor}{RGB}{248,248,248}
\newenvironment{Shaded}{\begin{snugshade}}{\end{snugshade}}
\newcommand{\AlertTok}[1]{\textcolor[rgb]{0.94,0.16,0.16}{#1}}
\newcommand{\AnnotationTok}[1]{\textcolor[rgb]{0.56,0.35,0.01}{\textbf{\textit{#1}}}}
\newcommand{\AttributeTok}[1]{\textcolor[rgb]{0.77,0.63,0.00}{#1}}
\newcommand{\BaseNTok}[1]{\textcolor[rgb]{0.00,0.00,0.81}{#1}}
\newcommand{\BuiltInTok}[1]{#1}
\newcommand{\CharTok}[1]{\textcolor[rgb]{0.31,0.60,0.02}{#1}}
\newcommand{\CommentTok}[1]{\textcolor[rgb]{0.56,0.35,0.01}{\textit{#1}}}
\newcommand{\CommentVarTok}[1]{\textcolor[rgb]{0.56,0.35,0.01}{\textbf{\textit{#1}}}}
\newcommand{\ConstantTok}[1]{\textcolor[rgb]{0.00,0.00,0.00}{#1}}
\newcommand{\ControlFlowTok}[1]{\textcolor[rgb]{0.13,0.29,0.53}{\textbf{#1}}}
\newcommand{\DataTypeTok}[1]{\textcolor[rgb]{0.13,0.29,0.53}{#1}}
\newcommand{\DecValTok}[1]{\textcolor[rgb]{0.00,0.00,0.81}{#1}}
\newcommand{\DocumentationTok}[1]{\textcolor[rgb]{0.56,0.35,0.01}{\textbf{\textit{#1}}}}
\newcommand{\ErrorTok}[1]{\textcolor[rgb]{0.64,0.00,0.00}{\textbf{#1}}}
\newcommand{\ExtensionTok}[1]{#1}
\newcommand{\FloatTok}[1]{\textcolor[rgb]{0.00,0.00,0.81}{#1}}
\newcommand{\FunctionTok}[1]{\textcolor[rgb]{0.00,0.00,0.00}{#1}}
\newcommand{\ImportTok}[1]{#1}
\newcommand{\InformationTok}[1]{\textcolor[rgb]{0.56,0.35,0.01}{\textbf{\textit{#1}}}}
\newcommand{\KeywordTok}[1]{\textcolor[rgb]{0.13,0.29,0.53}{\textbf{#1}}}
\newcommand{\NormalTok}[1]{#1}
\newcommand{\OperatorTok}[1]{\textcolor[rgb]{0.81,0.36,0.00}{\textbf{#1}}}
\newcommand{\OtherTok}[1]{\textcolor[rgb]{0.56,0.35,0.01}{#1}}
\newcommand{\PreprocessorTok}[1]{\textcolor[rgb]{0.56,0.35,0.01}{\textit{#1}}}
\newcommand{\RegionMarkerTok}[1]{#1}
\newcommand{\SpecialCharTok}[1]{\textcolor[rgb]{0.00,0.00,0.00}{#1}}
\newcommand{\SpecialStringTok}[1]{\textcolor[rgb]{0.31,0.60,0.02}{#1}}
\newcommand{\StringTok}[1]{\textcolor[rgb]{0.31,0.60,0.02}{#1}}
\newcommand{\VariableTok}[1]{\textcolor[rgb]{0.00,0.00,0.00}{#1}}
\newcommand{\VerbatimStringTok}[1]{\textcolor[rgb]{0.31,0.60,0.02}{#1}}
\newcommand{\WarningTok}[1]{\textcolor[rgb]{0.56,0.35,0.01}{\textbf{\textit{#1}}}}
\usepackage{longtable,booktabs}
\usepackage{graphicx,grffile}
\makeatletter
\def\maxwidth{\ifdim\Gin@nat@width>\linewidth\linewidth\else\Gin@nat@width\fi}
\def\maxheight{\ifdim\Gin@nat@height>\textheight\textheight\else\Gin@nat@height\fi}
\makeatother
% Scale images if necessary, so that they will not overflow the page
% margins by default, and it is still possible to overwrite the defaults
% using explicit options in \includegraphics[width, height, ...]{}
\setkeys{Gin}{width=\maxwidth,height=\maxheight,keepaspectratio}
\IfFileExists{parskip.sty}{%
\usepackage{parskip}
}{% else
\setlength{\parindent}{0pt}
\setlength{\parskip}{6pt plus 2pt minus 1pt}
}
\setlength{\emergencystretch}{3em}  % prevent overfull lines
\providecommand{\tightlist}{%
  \setlength{\itemsep}{0pt}\setlength{\parskip}{0pt}}
\setcounter{secnumdepth}{5}
% Redefines (sub)paragraphs to behave more like sections
\ifx\paragraph\undefined\else
\let\oldparagraph\paragraph
\renewcommand{\paragraph}[1]{\oldparagraph{#1}\mbox{}}
\fi
\ifx\subparagraph\undefined\else
\let\oldsubparagraph\subparagraph
\renewcommand{\subparagraph}[1]{\oldsubparagraph{#1}\mbox{}}
\fi

%%% Use protect on footnotes to avoid problems with footnotes in titles
\let\rmarkdownfootnote\footnote%
\def\footnote{\protect\rmarkdownfootnote}

%%% Change title format to be more compact
\usepackage{titling}

% Create subtitle command for use in maketitle
\providecommand{\subtitle}[1]{
  \posttitle{
    \begin{center}\large#1\end{center}
    }
}

\setlength{\droptitle}{-2em}

  \title{Using R at Grattan Institute}
    \pretitle{\vspace{\droptitle}\centering\huge}
  \posttitle{\par}
    \author{Will Mackey and Matt Cowgill}
    \preauthor{\centering\large\emph}
  \postauthor{\par}
      \predate{\centering\large\emph}
  \postdate{\par}
    \date{2019-08-01}

\usepackage{booktabs}
\usepackage{amsthm}
\makeatletter
\def\thm@space@setup{%
  \thm@preskip=8pt plus 2pt minus 4pt
  \thm@postskip=\thm@preskip
}
\makeatother

\begin{document}
\maketitle

{
\setcounter{tocdepth}{1}
\tableofcontents
}
\hypertarget{introduction}{%
\chapter*{Introduction}\label{introduction}}
\addcontentsline{toc}{chapter}{Introduction}

R is good and cool. Do you want to be good and cool? Use R!

\hypertarget{intro}{%
\chapter{Using R at Grattan}\label{intro}}

\begin{Shaded}
\begin{Highlighting}[]
\KeywordTok{library}\NormalTok{(tidyverse)}
\end{Highlighting}
\end{Shaded}

This document sets out good practices for structuring your R analysis at Grattan Institute. Having a clear, consistent structure for our analyses means that our work is more easily checked and revised, including by ourselves in the future. A small investment of time up front to set up your analysis will save time (your own and others') down the track.

This guide is designed for \emph{everyone} using R at Grattan. It includes a combination of rules and guidelines.

You should also be aware of the Grattan Institute R Style Guide, which lives in the same place as this document.

Any compaints or comments about this guide can be sent to Will or Matt, respectively.

\hypertarget{why-use-r}{%
\section{Why use R?}\label{why-use-r}}

It's good and cool!

\hypertarget{using-r-projects-for-a-fully-reproducible-workflow.}{%
\section{Using R projects for a fully reproducible workflow.}\label{using-r-projects-for-a-fully-reproducible-workflow.}}

\emph{Finally adhering to the `hit by a bus' rule.}

Cover:
1. setwd() and machine-speficic filepaths are bad
2. relative file paths are good
3. RStudio projects are an easy, reproducible way to set your wd

\hypertarget{filepaths}{%
\subsection{Filepaths}\label{filepaths}}

Filepaths should be relative to the working directory, and the working directory should be set by the project.

\textbf{Good}

\begin{Shaded}
\begin{Highlighting}[]
\NormalTok{hes <-}\StringTok{ }\KeywordTok{read_csv}\NormalTok{(}\StringTok{"data/HES/hes1516.csv"}\NormalTok{)}
\KeywordTok{grattan_save}\NormalTok{(}\StringTok{"images/expenditure_by_income.pdf"}\NormalTok{)}
\end{Highlighting}
\end{Shaded}

\textbf{Bad}

\begin{Shaded}
\begin{Highlighting}[]
\NormalTok{hes <-}\StringTok{ }\KeywordTok{read_csv}\NormalTok{(}\StringTok{"/Users/mcowgill/Desktop/hes1516.csv"}\NormalTok{)}
\NormalTok{hes <-}\StringTok{ }\KeywordTok{read_csb}\NormalTok{(}\StringTok{"C:\textbackslash{}Users\textbackslash{}mcowgill\textbackslash{}Desktop\textbackslash{}hes1516.csv"}\NormalTok{)}
\KeywordTok{grattan_save}\NormalTok{(}\StringTok{"/Users/mcowgill/Desktop/images/expenditure_by_income.pdf"}\NormalTok{)}
\end{Highlighting}
\end{Shaded}

\hypertarget{keep-your-scripts-manageable}{%
\subsection{Keep your scripts manageable}\label{keep-your-scripts-manageable}}

As a general rule of thumb, use one script per output. It should be clear what your script is trying to do (use comments!).

Consider breaking your analysis into pieces. For example:

\begin{itemize}
\tightlist
\item
  01\_import.R
\item
  02\_tidy.R
\item
  03\_model.R
\item
  04\_visualise.R
\end{itemize}

\textbf{Don't} include interactive work (like \texttt{View(mydf)}, \texttt{str(mydf)}, \texttt{mean(mydf\$variable)}, etc.) in your saved script.

\hypertarget{use-subfolders-of-your-project-folder}{%
\subsection{Use subfolders of your project folder}\label{use-subfolders-of-your-project-folder}}

Remember the hit-by-a-bus rule. It should be easy for any Grattan colleague to open your project folder and get up to speed with what it does. Putting all your files - raw data, scripts, output - in the one folder makes it harder to understand how your work fits together.

Use subfolders to clearly separate your code, raw data, and output.

\hypertarget{grattan-coding-style-guide}{%
\section{Grattan coding style guide}\label{grattan-coding-style-guide}}

Short summary of why

Link to style guide

\hypertarget{what-is-the-tidyverse-and-why-do-we-use-it}{%
\section{What is the tidyverse and why do we use it?}\label{what-is-the-tidyverse-and-why-do-we-use-it}}

Introduce following chapters

\hypertarget{an-introduction-to-rmarkdown}{%
\section{An introduction to RMarkdown}\label{an-introduction-to-rmarkdown}}

\hypertarget{resources-in-this-package}{%
\section{Resources in this package}\label{resources-in-this-package}}

\begin{itemize}
\tightlist
\item
  Starting a piece of analysis `cheat sheet'.
\item
  Updated style guide.
\item
  Written guide/slides.
\end{itemize}

\hypertarget{data-visualisation}{%
\chapter{Data Visualisation}\label{data-visualisation}}

\hypertarget{using-ggplot2-to-create-graphs-in-r}{%
\section{\texorpdfstring{Using \texttt{ggplot2} to create graphs in R}{Using ggplot2 to create graphs in R}}\label{using-ggplot2-to-create-graphs-in-r}}

\hypertarget{concepts}{%
\subsection{Concepts}\label{concepts}}

Main ingredients to a \texttt{ggplot} chart:
- Tidy data
- Aesthetics
- Geoms

Along with:
- Facets
- Colours
- Labels

\hypertarget{bar-charts}{%
\subsection{Bar charts}\label{bar-charts}}

\texttt{geom\_bar} if you have unit-record data and you want the geom to calculate something (count, sum, etc).
\texttt{geom\_col} if you want to plot numbers exactly as they are. This is how charts in Excel or Powerpoint work. It is the same as \texttt{geom\_bar(stat\ =\ "identity")}

The \texttt{position} argument\ldots{}

Remember: don't use too many colours (\ldots and other viz tips from the Chart Style Guide)

\hypertarget{line-charts}{%
\subsection{Line charts}\label{line-charts}}

\hypertarget{scatter-plots}{%
\subsection{Scatter plots}\label{scatter-plots}}

\texttt{geom\_point}
\texttt{geom\_smooth}

\hypertarget{distributions}{%
\subsection{Distributions}\label{distributions}}

\texttt{geom\_histogram}
\texttt{geom\_density}

\texttt{ggridges::}

\hypertarget{maps}{%
\subsection{Maps}\label{maps}}

\texttt{absmapsdata}

\hypertarget{using-grattantheme-to-make-and-export-grattan-y-charts}{%
\section{Using grattantheme to make and export ``Grattan-y'' charts}\label{using-grattantheme-to-make-and-export-grattan-y-charts}}

(current text taken from an email I sent an intern once -- will need to be updated)

The \texttt{grattantheme} package is hosted here: \url{https://github.com/mattcowgill/grattantheme}

You can install it with install\_github from the \texttt{remotes} package:

\begin{Shaded}
\begin{Highlighting}[]
\KeywordTok{install.packages}\NormalTok{(}\StringTok{"remotes"}\NormalTok{)}
\NormalTok{remotes}\OperatorTok{::}\KeywordTok{install_github}\NormalTok{(}\StringTok{"mattcowgill/grattantheme"}\NormalTok{)}

\KeywordTok{library}\NormalTok{(grattantheme)}
\end{Highlighting}
\end{Shaded}

You can look at explanatory vignette with \texttt{vignette("using\_grattantheme",\ "grattantheme")}.

Key functions are:

\texttt{theme\_grattan()} to set the size and default single-colour. Set flipped = TRUE if you have used coord\_flip() on the chart.

\texttt{grattan\_fill\_manual(n\ =\ 2)} to manually set the fill colours of a chart if you have mapped fill to an aesthetic, eg \texttt{aes(\ldots{},\ fill\ =\ gender)}. Set \texttt{n} to the number of colours you have.

\texttt{grattan\_colour\_manual(n\ =\ 2)} to manually set the colour if you have mapped colour to an aesthetic, eg \texttt{aes(\ldots{},\ colour\ =\ gender)}.

\texttt{grattan\_y\_continuous()} to properly style the Y-axis and align it with zero.

\texttt{grattan\_save()} instead of \texttt{ggsave()} to export charts.

\hypertarget{creating-simple-interactive-graphs-with-plotly}{%
\section{\texorpdfstring{Creating simple interactive graphs with \texttt{plotly}}{Creating simple interactive graphs with plotly}}\label{creating-simple-interactive-graphs-with-plotly}}

\texttt{plotly::ggplotly()}

\hypertarget{reading-data}{%
\chapter{Reading data}\label{reading-data}}

\hypertarget{importing-data}{%
\section{Importing data}\label{importing-data}}

\hypertarget{reading-csv-files}{%
\subsection{Reading CSV files}\label{reading-csv-files}}

\hypertarget{read_csv}{%
\subsubsection{\texorpdfstring{\texttt{read\_csv()}}{read\_csv()}}\label{read_csv}}

The \texttt{read\_csv()} function from the \texttt{tidyverse} is quicker and smarter than \texttt{read.csv} in base R.

Pitfalls:
1. read\_csv is quicker because it surveys a sample of the data

We can also compress \texttt{.csv} files into \texttt{.zip} files and read them \emph{directly} using \texttt{read\_csv()}:

\begin{Shaded}
\begin{Highlighting}[]
\KeywordTok{read_csv}\NormalTok{(}\StringTok{"data/my_data.zip"}\NormalTok{)}
\end{Highlighting}
\end{Shaded}

This is useful for two reasons:

\begin{enumerate}
\def\labelenumi{\arabic{enumi}.}
\tightlist
\item
  The data takes up less room on your computer; and
\item
  The original data, which shouldn't ever be directly edited, is protected and cannot be directly edited.
\end{enumerate}

\hypertarget{data.tablefread}{%
\subsubsection{\texorpdfstring{\texttt{data.table::fread()}}{data.table::fread()}}\label{data.tablefread}}

The \texttt{fread} function from \texttt{data.table} is quicker than both \texttt{read.csv} and \texttt{read\_csv}.

\hypertarget{readxlread_excel}{%
\subsection{\texorpdfstring{\texttt{readxl::read\_excel()}}{readxl::read\_excel()}}\label{readxlread_excel}}

\hypertarget{rio}{%
\subsection{\texorpdfstring{\texttt{rio}}{rio}}\label{rio}}

\hypertarget{readabs}{%
\subsection{\texorpdfstring{\texttt{readabs}}{readabs}}\label{readabs}}

\hypertarget{reading-common-files}{%
\section{Reading common files:}\label{reading-common-files}}

\begin{itemize}
\tightlist
\item
  TableBuilder CSVSTRINGs
\item
  HES household file
\item
  SIH
\item
  LSAY and derivatives
\end{itemize}

See data directory for a list of microdata available to Grattan.

\hypertarget{appropriately-renaming-variables}{%
\section{Appropriately renaming variables}\label{appropriately-renaming-variables}}

As shown in the style guide

Add \texttt{rename\_abs} function to a common Grattan package?

\hypertarget{getting-to-tidy-data}{%
\section{Getting to tidy data}\label{getting-to-tidy-data}}

\texttt{pivot\_long()} and \texttt{pivot\_wide()}
\emph{Make sure these are stable btw}

\hypertarget{different-data-types}{%
\chapter{Different data types}\label{different-data-types}}

\hypertarget{tidy-data}{%
\section{Tidy data}\label{tidy-data}}

Other data structures

\hypertarget{dates-with-lubridate}{%
\section{\texorpdfstring{Dates with \texttt{lubridate::}}{Dates with lubridate::}}\label{dates-with-lubridate}}

The \texttt{lubridate::} package

\hypertarget{strings-with-stringr}{%
\section{\texorpdfstring{Strings with \texttt{stringr::}}{Strings with stringr::}}\label{strings-with-stringr}}

\begin{itemize}
\tightlist
\item
  Replacing values
\item
  Matching values
\item
  Separating columns
\end{itemize}

\hypertarget{factors-with-forcats}{%
\section{\texorpdfstring{Factors with \texttt{forcats::}}{Factors with forcats::}}\label{factors-with-forcats}}

\begin{itemize}
\tightlist
\item
  Dangers with factors
\end{itemize}

\hypertarget{data-transformation}{%
\chapter{Data transformation}\label{data-transformation}}

\hypertarget{the-pipe}{%
\section{The pipe}\label{the-pipe}}

\hypertarget{key-dplyr-functions}{%
\section{\texorpdfstring{Key \texttt{dplyr} functions:}{Key dplyr functions:}}\label{key-dplyr-functions}}

All have the same syntax structure, which enable pipe-chains.

\hypertarget{filter-with-filter}{%
\section{\texorpdfstring{Filter with \texttt{filter()}}{Filter with filter()}}\label{filter-with-filter}}

\hypertarget{arrange-with-arrange}{%
\section{\texorpdfstring{Arrange with \texttt{arrange()}}{Arrange with arrange()}}\label{arrange-with-arrange}}

\hypertarget{select-variables-with-select}{%
\section{\texorpdfstring{Select variables with \texttt{select()}}{Select variables with select()}}\label{select-variables-with-select}}

\hypertarget{group-data-with-group_by}{%
\section{\texorpdfstring{Group data with \texttt{group\_by()}}{Group data with group\_by()}}\label{group-data-with-group_by}}

\hypertarget{edit-and-add-new-variables-with-mutate}{%
\section{\texorpdfstring{Edit and add new variables with \texttt{mutate()}}{Edit and add new variables with mutate()}}\label{edit-and-add-new-variables-with-mutate}}

\hypertarget{cases-when-you-should-use-case_when}{%
\subsection{\texorpdfstring{Cases when you should use \texttt{case\_when()}}{Cases when you should use case\_when()}}\label{cases-when-you-should-use-case_when}}

\hypertarget{summarise-data-with-summarise}{%
\section{\texorpdfstring{Summarise data with \texttt{summarise()}}{Summarise data with summarise()}}\label{summarise-data-with-summarise}}

\hypertarget{joining-datasets-with-_join}{%
\section{\texorpdfstring{Joining datasets with \texttt{*\_join()}}{Joining datasets with *\_join()}}\label{joining-datasets-with-_join}}

\hypertarget{analysis}{%
\chapter{Analysis}\label{analysis}}

\hypertarget{creating-functions}{%
\chapter{Creating functions}\label{creating-functions}}

\hypertarget{it-can-be-useful-to-make-your-own-function}{%
\section{It can be useful to make your own function}\label{it-can-be-useful-to-make-your-own-function}}

Why on earth would you create your own function?

\hypertarget{defining-simple-functions}{%
\section{Defining simple functions}\label{defining-simple-functions}}

\hypertarget{more-complex-functions}{%
\section{More complex functions}\label{more-complex-functions}}

\hypertarget{sets-of-functions}{%
\section{Sets of functions}\label{sets-of-functions}}

\hypertarget{using-purrrmap}{%
\section{\texorpdfstring{Using \texttt{purrr::map}}{Using purrr::map}}\label{using-purrrmap}}

\hypertarget{sharing-your-useful-functions-with-grattan}{%
\section{Sharing your useful functions with Grattan}\label{sharing-your-useful-functions-with-grattan}}

\hypertarget{version-control}{%
\chapter{Version control}\label{version-control}}

\hypertarget{version-control-is-important-and-intimidating}{%
\section{Version control is important and intimidating}\label{version-control-is-important-and-intimidating}}

Version control is great!

\hypertarget{github}{%
\section{Github}\label{github}}

We use Github to version-control and share reports in LaTeX, so you're already a bit set-up.

\hypertarget{git}{%
\section{Git}\label{git}}

Using Git within R Studio\ldots{}

\bibliography{book.bib,packages.bib}


\end{document}
